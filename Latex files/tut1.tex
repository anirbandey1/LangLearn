\documentclass[10pt]{article}
\usepackage{amsmath}
%\usepackage{ulem}
\begin{document}
	\begin{center}
		{\Huge text at the center}
	\end{center}

	I hope this is okay
	\textbf{Bold}
	\textit{Italic}
	\underline{Underlined}
	\textbf{\textit{\underline{combo}}}
	
	%\uline{The country}	
	
	\emph{Emphasis should be on}
	
	%font family
	
	\textrm{Roman}
	\textsf{Sans Serif} \\
	\texttt{Typewriter text}
	
	%text justification/alignment
	\begin{center}
		Center justified
	\end{center}
	\begin{flushleft}
		Left jusified
	\end{flushleft}
	\begin{flushright}
		Right Justified
	\end{flushright}

	%indentation
	%\set length
	
	
	%math manipulation
	%display style
	%inline mode
	
	\[ f(x) = x^2+y^{-1}\]
	
	%to aliign a set of equations
	% ampersand symbol tells how to align
	\begin{align*}
		f(x) &= a_2 x^2 + a_1 x +a_0 \\
			&= 3 x^2 -5 x + 0 
	\end{align*}

	\begin{align}
		f(x) &= a_2 x^2 + a_1 x +a_0 
		\nonumber \\
		&= 3 x^2 -5 x + 0 
	\end{align}

	The mathematical function is 
	\(f(x)= x^2 -1\) \\
	The problem is
	$g(y)= \sqrt[3]{y+1}$
	
	\[ 
		\sum_{n=1}^{5} 2^{-n} = \frac{3}{6}
	\]
	
	% force displaystylemath in inline mode
	% \[ \displaystyle\]
	
	%force inline style
	% \( \textstyle\)
	
	%Arithmetic
	1+1 
	
	1-2
	1$\cdot$2 
	
	1$\div$2 
	
	$\frac{36}{58}$
	2452$\tfrac{54}{24}$5252 \\
	%\dfrac{452}{53}
	
	%Superscript/Subscript
	\(a_2 x^2 + \) 
	% use brackets for grouping 
	\(a_{-2}\)  
	
	Simultaneous superscript and subscript
	\(a^2_0\)
	\(a_0^2\)
	
	%Parenthesis
	
	%size not proper
	\[
	(\sum_{n=0}^{\infty})
	\]
	%proper size
	\[
	\left(\sum_{n=0}^{\infty}\right)
	\]
	%different sizes
	\[
		(a)
		\big( a \big)
		\Big( a \Big)
		\bigg( a \bigg)
		\Bigg[ a \Bigg]
	\]]
	
	%text in math mode
	%mistake no white space
	\[
		n = ab \text{where} a \text{and} b \text{are natural numbers}
	\]
	%correct with whitespace
	\[
	n = ab \text{ where } a \text{ and } b \text{ are natural numbers}
	\]
	
	%Greek letters
	\[
		\alpha \cdot \Gamma
	\]
	
	%AMS black board font for defining sets
	
	%\[ \mathbb{N}	\]
\end{document}